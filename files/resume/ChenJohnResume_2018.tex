% LaTeX resume using res.cls
\documentclass[margin]{res}
\usepackage{palatino}
\usepackage{tabto}
\setlength{\textwidth}{5.1in} % set width of text portion

\begin{document}
	
% Center the name over the entire width of resume:
\moveleft.5\hoffset\centerline{\Large\bf John Matthew Chen}
% Draw a horizontal line the whole width of resume:
\moveleft\hoffset\vbox{\hrule width\resumewidth height 1pt}

\begin{minipage}{0.4\textwidth}
\moveleft\hoffset\vbox{
PH\#3-221 Queen Street South \\ 
Kitchener, ON N2G 1W5}
\end{minipage}
\hfill
\begin{minipage}{0.6\textwidth}
\flushright\moveright\hoffset\vbox{
jmc\_tt@live.com \\
(204) 807-4524 \\
https://johntt.github.io/
}
\end{minipage}

\begin{resume}

\section{WORK \\ EXPERIENCE} 
{\sl Embedded Software Developer} \hfill August 2018 - Present \\
Aeryon Labs Inc., Waterloo, Ontario

{\sl Technician} \hfill July 2018 \\
Mobile Klinik, Kildonan Place, Winnipeg, Manitoba
\begin{itemize} \itemsep -0pt 
	\item Performed screen replacements for customers with broken iPhone 6 and 7 models within advertised repair estimates of about an hour.
\end{itemize}

{\sl Research Assistant (Advisor: Dr. Carl Ho)} \hfill May - August, 2017 \\
RIGA Lab - University of Manitoba
\begin{itemize} \itemsep -0pt 
	\item Designed and simulated a programmable DC-AC grid-tied, active load using the Plexim PLECS power electronics software.
	\item Programmed the TI LAUNCHXL-F28377S using the C programming language and the Code Composer Studio IDE to implement PI (Proportional-Integral) and hysteresis controllers.
\end{itemize}

{\sl Research Assistant (Advisor: Dr. Douglas Buchanan)} \hfill May - August, 2016 \\
Nano Systems Research Lab - University of Manitoba
\begin{itemize} \itemsep -0pt 
	\item Performed research into micro-machined (MEMS) ultrasound transducers with the characteristics needed for Synthetic Aperture (SA) imaging.
	\item Designed a graphical user interface in LabVIEW which allows the user to control an Optics Focus Motorized XYZ Stage connected over RS232, and save measurements recorded by an oscilloscope connected over GPIB.
\end{itemize}

\section{EDUCATION} 
Bachelor of Science in Electrical Engineering \hfill May 2018 \\
University of Manitoba, Winnipeg, MB, Canada \\	
Grade Point Average: 4.09/4.50 

\section{ENGINEERING \\ PROJECTS} 

{\bf University of Manitoba ecoMotion (Shell Eco-Marathon) Team} \hfill 2016 - 2018 \\
{\sl Secretary (2016-2018) and Propulsion Controls Lead (2017-2018)}
\begin{itemize}  \itemsep -0pt %reduce space between items
	\item Executive member of student design team, UM ecoMotion, founded in January 2016, which competes annually in the Shell Eco-Marathon competition.
	\item Placed 14th (80.62 km/kWh) in 2017, and 10th (101.8 km/kWh) in 2018, in the Americas Battery Electric Prototype vehicle category.	
	\item Created a SPI software driver to draw shapes, numbers and letters onto the Pervasive Displays 4.41″ E-Paper Display.
	\item Programmed STM32 Nucleo boards in C using SW4STM32 (Eclipse-based IDE) to parse SAE J1939 CAN messages from the AllCell Battery Management System and to send CAN messages to the motor controller.
	\item Utilized an open-source Git repository for the VESC STM32 F4 based BLDC motor controller running ChibiOS for the propulsion system.
	
	\newpage
	\moveleft.5\hoffset\centerline{\Large\bf John Matthew Chen}
	\moveleft\hoffset\vbox{\hrule width\resumewidth height 1pt}	
	
	\item Ported a C++ Arduino library to STM32 for an Adafruit I2C Seven-Segment display which was used as the speedometer.
	\item Maintained version control of three different projects for the throttle, master and display Nucleo boards using GitLab.
	\item Proficient in using test equipment, such as multi-meters, oscilloscopes, and logic analyzers to debug issues with protocols (CAN, I2C, SPI, etc.).
	\item Experience using hand tools (wrenches, hacksaws, screwdrivers) and power tools (drills, jigsaws, foam cutters) to manufacture parts for the vehicle.
	\item Designed schematics and layout files for a 48V/20A Joulemeter using Altium Designer based off the Teensy 3.2.
	\item Led the registration process for the team resulting in over 60 members and over \$2500 in membership fees for the 2016-2018 period.
	\item Created and designed the team website, http://umecomotion.ca/, using the Drupal content-management framework.
	\item Obtained \$8000 (2016) and \$6000 (2017) in sponsorship from the Engineering Endowment Fund.

\end{itemize}

{\bf International Olympiad in Informatics} \hfill 2010, 2012, 2013

{\bf Capstone Design Project - 3rd Place Award} \hfill April 2018 \\
{\sl IEEE Canada, Winnipeg Section}
\begin{itemize}  \itemsep -0pt 
	\item Designed a Bluetooth LE glove based on the Adafruit nRF52 Feather and the BNO055 IMU that converts user hand movements and gestures into mouse commands.
	\item Modified an existing UWP Microsoft application for demonstration purposes using Visual Studio, C\# and XAML.
\end{itemize}

{\bf Relevant Coursework}
\begin{itemize}  \itemsep -0pt 
	\item Completed the courses: Power Electronics, High Voltage Engineering, Control Systems, Design of RF/Wireless Systems, Microwave Engineering, Antennas, Digital Communications, Signal Processing 1\&2
	\item Experience using both Python and MATLAB for file I/O, calculations, and graphing.
\end{itemize}
	
\section{COMPUTER \\ SKILLS} 
{\sl Programming Languages:} C, C++, C\#, Python 2/3, Java, Javascript, Pascal, HTML, XML, XAML, CSS, Verilog HDL, 68HC12 Assembly, LaTeX, SQL

{\sl Software:} Eclipse, MPLAB X, TI CCS, SW4STM32, Visual Studio, .NET Framework, Universal Windows Platform (UWP), Arduino, Code::Blocks, Git,  mbed, Altium Designer, Drupal, cPanel, MATLAB, NI LabVIEW, AutoCAD, NI Multisim, PSCAD, Plexim PLECS, COMSOL Multiphysics, Advanced Design System (ADS), FEKO, OriginLab Origin, Tanner L-Edit, Android Studio, Microsoft Office (Word, Excel, Powerpoint, Access, Project, Visio)

\section{ACADEMIC \\ AWARDS} 
Engineering Academic Excellence Award - First Place Electrical \hfill September 2016 \\
Undergraduate Research Award \hfill April 2016, April 2017 \\
Price Industries Limited Entrance Scholarships for Engineering \hfill March 2015 
           
\iffalse
{\sl International Olympiad in Informatics}
\begin{itemize}  \itemsep -0pt 
	\item Competed in an annual programming competition which brings together exceptionally talented pupils from over 80 countries
\end{itemize} 
\begin{enumerate}  \itemsep -0pt %reduce space between items
	\item University of Waterloo, Ontario, Canada \hfill August 14-21, 2010
	\item Sirmione - Montichiari, Italy \hfill September 23-30, 2012
	\item University of Queensland, Brisbane, Australia \hfill July 6-13, 2013
\end{enumerate}
\fi

\end{resume}
\end{document}




